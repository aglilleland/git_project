% Options for packages loaded elsewhere
\PassOptionsToPackage{unicode}{hyperref}
\PassOptionsToPackage{hyphens}{url}
%
\documentclass[
  norsk,
]{article}
\usepackage{lmodern}
\usepackage{amssymb,amsmath}
\usepackage{ifxetex,ifluatex}
\ifnum 0\ifxetex 1\fi\ifluatex 1\fi=0 % if pdftex
  \usepackage[T1]{fontenc}
  \usepackage[utf8]{inputenc}
  \usepackage{textcomp} % provide euro and other symbols
\else % if luatex or xetex
  \usepackage{unicode-math}
  \defaultfontfeatures{Scale=MatchLowercase}
  \defaultfontfeatures[\rmfamily]{Ligatures=TeX,Scale=1}
\fi
% Use upquote if available, for straight quotes in verbatim environments
\IfFileExists{upquote.sty}{\usepackage{upquote}}{}
\IfFileExists{microtype.sty}{% use microtype if available
  \usepackage[]{microtype}
  \UseMicrotypeSet[protrusion]{basicmath} % disable protrusion for tt fonts
}{}
\makeatletter
\@ifundefined{KOMAClassName}{% if non-KOMA class
  \IfFileExists{parskip.sty}{%
    \usepackage{parskip}
  }{% else
    \setlength{\parindent}{0pt}
    \setlength{\parskip}{6pt plus 2pt minus 1pt}}
}{% if KOMA class
  \KOMAoptions{parskip=half}}
\makeatother
\usepackage{xcolor}
\IfFileExists{xurl.sty}{\usepackage{xurl}}{} % add URL line breaks if available
\IfFileExists{bookmark.sty}{\usepackage{bookmark}}{\usepackage{hyperref}}
\hypersetup{
  pdftitle={Reproduserbarhet i sammenheng med Stordata},
  pdfauthor={Anne Grethe Lilleland, Arian Steen},
  pdflang={nb-no},
  hidelinks,
  pdfcreator={LaTeX via pandoc}}
\urlstyle{same} % disable monospaced font for URLs
\usepackage[margin=1in]{geometry}
\usepackage{color}
\usepackage{fancyvrb}
\newcommand{\VerbBar}{|}
\newcommand{\VERB}{\Verb[commandchars=\\\{\}]}
\DefineVerbatimEnvironment{Highlighting}{Verbatim}{commandchars=\\\{\}}
% Add ',fontsize=\small' for more characters per line
\usepackage{framed}
\definecolor{shadecolor}{RGB}{248,248,248}
\newenvironment{Shaded}{\begin{snugshade}}{\end{snugshade}}
\newcommand{\AlertTok}[1]{\textcolor[rgb]{0.94,0.16,0.16}{#1}}
\newcommand{\AnnotationTok}[1]{\textcolor[rgb]{0.56,0.35,0.01}{\textbf{\textit{#1}}}}
\newcommand{\AttributeTok}[1]{\textcolor[rgb]{0.77,0.63,0.00}{#1}}
\newcommand{\BaseNTok}[1]{\textcolor[rgb]{0.00,0.00,0.81}{#1}}
\newcommand{\BuiltInTok}[1]{#1}
\newcommand{\CharTok}[1]{\textcolor[rgb]{0.31,0.60,0.02}{#1}}
\newcommand{\CommentTok}[1]{\textcolor[rgb]{0.56,0.35,0.01}{\textit{#1}}}
\newcommand{\CommentVarTok}[1]{\textcolor[rgb]{0.56,0.35,0.01}{\textbf{\textit{#1}}}}
\newcommand{\ConstantTok}[1]{\textcolor[rgb]{0.00,0.00,0.00}{#1}}
\newcommand{\ControlFlowTok}[1]{\textcolor[rgb]{0.13,0.29,0.53}{\textbf{#1}}}
\newcommand{\DataTypeTok}[1]{\textcolor[rgb]{0.13,0.29,0.53}{#1}}
\newcommand{\DecValTok}[1]{\textcolor[rgb]{0.00,0.00,0.81}{#1}}
\newcommand{\DocumentationTok}[1]{\textcolor[rgb]{0.56,0.35,0.01}{\textbf{\textit{#1}}}}
\newcommand{\ErrorTok}[1]{\textcolor[rgb]{0.64,0.00,0.00}{\textbf{#1}}}
\newcommand{\ExtensionTok}[1]{#1}
\newcommand{\FloatTok}[1]{\textcolor[rgb]{0.00,0.00,0.81}{#1}}
\newcommand{\FunctionTok}[1]{\textcolor[rgb]{0.00,0.00,0.00}{#1}}
\newcommand{\ImportTok}[1]{#1}
\newcommand{\InformationTok}[1]{\textcolor[rgb]{0.56,0.35,0.01}{\textbf{\textit{#1}}}}
\newcommand{\KeywordTok}[1]{\textcolor[rgb]{0.13,0.29,0.53}{\textbf{#1}}}
\newcommand{\NormalTok}[1]{#1}
\newcommand{\OperatorTok}[1]{\textcolor[rgb]{0.81,0.36,0.00}{\textbf{#1}}}
\newcommand{\OtherTok}[1]{\textcolor[rgb]{0.56,0.35,0.01}{#1}}
\newcommand{\PreprocessorTok}[1]{\textcolor[rgb]{0.56,0.35,0.01}{\textit{#1}}}
\newcommand{\RegionMarkerTok}[1]{#1}
\newcommand{\SpecialCharTok}[1]{\textcolor[rgb]{0.00,0.00,0.00}{#1}}
\newcommand{\SpecialStringTok}[1]{\textcolor[rgb]{0.31,0.60,0.02}{#1}}
\newcommand{\StringTok}[1]{\textcolor[rgb]{0.31,0.60,0.02}{#1}}
\newcommand{\VariableTok}[1]{\textcolor[rgb]{0.00,0.00,0.00}{#1}}
\newcommand{\VerbatimStringTok}[1]{\textcolor[rgb]{0.31,0.60,0.02}{#1}}
\newcommand{\WarningTok}[1]{\textcolor[rgb]{0.56,0.35,0.01}{\textbf{\textit{#1}}}}
\usepackage{graphicx,grffile}
\makeatletter
\def\maxwidth{\ifdim\Gin@nat@width>\linewidth\linewidth\else\Gin@nat@width\fi}
\def\maxheight{\ifdim\Gin@nat@height>\textheight\textheight\else\Gin@nat@height\fi}
\makeatother
% Scale images if necessary, so that they will not overflow the page
% margins by default, and it is still possible to overwrite the defaults
% using explicit options in \includegraphics[width, height, ...]{}
\setkeys{Gin}{width=\maxwidth,height=\maxheight,keepaspectratio}
% Set default figure placement to htbp
\makeatletter
\def\fps@figure{htbp}
\makeatother
\setlength{\emergencystretch}{3em} % prevent overfull lines
\providecommand{\tightlist}{%
  \setlength{\itemsep}{0pt}\setlength{\parskip}{0pt}}
\setcounter{secnumdepth}{-\maxdimen} % remove section numbering
\ifxetex
  % Load polyglossia as late as possible: uses bidi with RTL langages (e.g. Hebrew, Arabic)
  \usepackage{polyglossia}
  \setmainlanguage[]{norsk}
\else
  \usepackage[shorthands=off,main=norsk]{babel}
\fi

\title{Reproduserbarhet i sammenheng med Stordata}
\usepackage{etoolbox}
\makeatletter
\providecommand{\subtitle}[1]{% add subtitle to \maketitle
  \apptocmd{\@title}{\par {\large #1 \par}}{}{}
}
\makeatother
\subtitle{Data Science Assignment HVL University}
\author{Anne Grethe Lilleland, Arian Steen}
\date{17/09/20}

\begin{document}
\maketitle

\hypertarget{introduksjon}{%
\subsection{\texorpdfstring{
\emph{Introduksjon}}{ Introduksjon}}\label{introduksjon}}

I denne oppgaven tar vi for oss viktigheten av reproduserbarhet når man
arbeider med stordata.Vi vil se på hva \emph{R} og \emph{R-studio} er,
og hvordan dette kan hjelpe oss med reproduserbarhet og replikerbarhet.
Vi vil også forklare hva \emph{R-markdown} og \emph{R-Notebook} er.

Utviklingen av diverse dataprogrammer (software) har styrket
analytikernes evne til å gjennomføre flerdimensjonal data-analyse. En
slik type analyse, gir muligheter til å gjennomføre komplekse
beregninger med svært raske resultater. Blant de ulike dataprogrammer
som brukes til dette finner man, for eksempel. SPSS,STATA,EXCEL og R.

Sistnevnte ``R'' tilbyr unike muligheter som skiller det fra andre
tilgjengelige statistiske verktøyene. Faktisk har R blitt hovedverktøyet
til data-analyse. De R ekspertene som bruker R-programmet, kalles
Data-Analyst eller Data Scietist, på grunn av det brede omfanget som
``R'' kan tilby.

I 1991 ble ``R'' opprettet av Ross Ihaka og Robert Gentleman i
``Department of Statistics ved University of Auckland''. I 1993 ble den
første kunngjøringen av ``R'' gjort for publikum. Ross og Roberts
erfaring med å utvikle ``R'' er dokumentert i en artikkel i det
vitenskapelige journal (Ihaka,R \& Gentleman,R 1996)

Populæriteten til ``R'' kommer fra diverse kilder. For det første er den
basert på multiplatform kode, som vil gjøre det mulig å bruke den på
tvers av ulike platformer (Dette fører til enklere sammarbeid mellom
flere parter) og åpen kildekode (Det sikrer reproduserbarheten av
analysen), det blir brukt av profesjonelle i arbeidslivet, store
selskaper og studenter som foreksempel (Økonomisk Data-analyse) brukt av
HVL studenter innen økonomifeltet. ``R'' gir oss muligheten til å jobbe
raskt og iretaivt med store datamengder.

For det andre, er brukervennligheten til ``R'' og det faktum at det
finnes store mengder læremateriale åpent,fritt og lett tilgjengelig på
nettet, gjør det mulig å reprodusere data, gjennomføre analyser og løse
diverse problemstillinger. ``R'' gjør det mulig for forskere å fange opp
og kommunisere detaljene i arbeidsflyten med større effektivitet. I
følgende paragrafer forteller vi nærmere om hvordan disse ulike
elementene øker vår mulighet til å gjennomføre en analyse ut ifra et
datasett ved bruk av ``R''.

\hypertarget{hvorfor-r-og-r-studio}{%
\subsubsection{\texorpdfstring{\textbf{Hvorfor R og R
studio?}}{Hvorfor R og R studio?}}\label{hvorfor-r-og-r-studio}}

Både R og R studio er åpen kildekode programvarer og er gratis. Det
brukes i sammenheng med forskning og datavisualisering. R er ett
kodespråk for statistisk databehandling og grafikk, mens R-studio er ett
integrert utviklingsmiljø for R. R og R studio gjør sammarbeid enklere
via feks. Git og Github. Man kan importere og manipulere data og kjøre
dette i ett maskinlæringsprogram via R studio. Skal man kjøre linerar
regrisjon eller lignende kan man bruke R og R studio. Analyser gjort i R
og R studio er reproduserbare, da alt som er gjort ligger inne i
``history'' og kan bli hentet frem igjen.

\hypertarget{historisk-data-arkivering}{%
\subsection{\texorpdfstring{\textbf{Historisk data
arkivering}}{Historisk data arkivering}}\label{historisk-data-arkivering}}

Historisk har data arkivering foregått på ark og store lagringslokaler,
arkivering på maskin begynte først da datamskiner ble kommersialisert og
gjort tilgjengelig for forskere. Men har nok begynt før det med IBM sine
mainframe maskiner hvor data ble lagret som var kritisk til å
oppretteholde sikkerheten til USA.

Dataarkivering har i det siste blitt effektivisert, hvor data har blitt
lagret på disker og kasetter som ikke eksisterte for noen tiår siden.

Da mekaniske deler ikke har evig levetid har dette blitt et problem, det
jobbes nå for å ta kopi og backup av all historisk data som vi har
samlet opp. Et eksempel er svalbard bunkeren hvor det er laget en
underjordisk bunker hvor kildekoden til noen av de mest kritiske
programmene vi har blir lagret, denne type data har blitt lagret på tape
for å forminske bit-loss.

De som jobber med data arkivering har sagt at det i gjennomsnitt tar 20
år før vi mister data fra standard magnetdrevne disker og opp imot 60 år
på cd-disker for eksempel, kommer ann på hvordan de er lagret,
arkivering på magnet-disker har sine egne problemer da de er spesielt
utsatt for sjokk og da ekstra utsatt for elektromagnetisk sjokk, fra
storm og uvær.

I det siste har prisene på SSDer gått ned, dette gir en unik mulighet
for å lagre data på en effektiv og plassbesparende måte hvor en kan
lagre store mengder på fysisk mindre disker. Dette gir oss en muligheten
til å ta backup av all data og spare både energi og tid.

\hypertarget{reproduserbarhet-og-replikerbarhet}{%
\subsection{\texorpdfstring{\emph{Reproduserbarhet og
replikerbarhet}}{Reproduserbarhet og replikerbarhet}}\label{reproduserbarhet-og-replikerbarhet}}

\hypertarget{hvorfor-bryr-vi-oss-om-reproduserbarhet-og-replikerbarhet}{%
\subsubsection{\texorpdfstring{\textbf{Hvorfor bryr vi oss om
reproduserbarhet og
replikerbarhet?}}{Hvorfor bryr vi oss om reproduserbarhet og replikerbarhet?}}\label{hvorfor-bryr-vi-oss-om-reproduserbarhet-og-replikerbarhet}}

Forskere prøver ofte å reprodusere eksprimenter for å øke troverdigheten
til forsøket og resultatene i eksprimentet. Hvis eksprimentet kan
reproduseres og få de samme svarene øker tiltroen/tillten til
resultatet.Reproduserbarhet går ut på å bruke de samme dataene som
allerede er brukt i forsåket, utføre da samme analysene, og ende opp med
de samme svarene.

Replikerbarhet blir sett på som ``gull'' standarden innenfor forskning.
Replikerbarhet går ut på at det skal være mulig å gjøre forsøket på
nytt, med nye data, og ende opp med samme svar. Reproduserbarhet er
nødvendig for å få replikerbarhet, men ikke tilstrekkelig nok til å gi
replikerbarhet.

Reproduserbarhet og replikerbarhet har vist seg og ikke være noen
selvfølge innen forskning. Det er flere meninger og tanker rundt dette.
Noen hevder forskningen burde være reproduserbar for at den skal kunne
publiseres. Andre går så lang og mener den også bør kunne replikeres.Vi
vil ikke gå nærmere inn på dette, men vi vil se mer på grunnene til at
studier ikke er reproduserbare og replikerbare, og ulike løsninger på
dette senere i oppgaven.

\hypertarget{reproduserbarhet-og-r-notebooks}{%
\subsection{\texorpdfstring{\textbf{Reproduserbarhet og
R-Notebooks}}{Reproduserbarhet og R-Notebooks}}\label{reproduserbarhet-og-r-notebooks}}

Repetering og intern reproduserbarhet et aspekt innen Data Science
feltet, og alle som håndterer og jobber/deler store mengder data. Det
handler om overensstemmelse mellom resultatene av målinger av samme
målstørrelse utført under samme målebetingelser (VIM, 1993).
Reproduserbarhet avspeiler i hvilken grad en har overenstemmelse i
gjentatte målinger på samme prøvemateriale, forutsatt at de utføres med
samme betingelser.

Dette er en viktig aspekt for å få sikre metodikken i en analyse. For å
måle graden av reproduserbarhet kan en benytte to eller flere
kontrollmaterialer (Som for eksempel resultat av en prøve) som dekker
måleområdet, og analysere det for eksempel 10 ganger, og se hvordan
resultatet endrer seg.

På denne måten kan man validere den gjennomførte analysen. Verdien av
hva forskjellen mellom to testresultater oppnådd under
reproduserbarhetsbetingelser kan forventes å oppstå i utgangpunk i
forskjellige konfidensintervall (sannsynlighet) som et intervallestimat.
For eksempel på omtrent 99\%, 95\% eller 90\%. Hvis vi bruker 95\%
konfidensintervall, vil det si at det er 95 prosent av tilfellene det
man utfører eksperimentet, vil populasjonens gjennomsnitt ligge innenfor
dette intervallet(Bergsaker, 2019).

\hypertarget{har-forskerne-incentiver-til-uxe5-vuxe6re-reproduserbare-eller-muxe5-de-tvinges}{%
\subsection{\texorpdfstring{\textbf{Har forskerne incentiver til å være
``Reproduserbare'' eller må de
tvinges?}}{Har forskerne incentiver til å være ``Reproduserbare'' eller må de tvinges?}}\label{har-forskerne-incentiver-til-uxe5-vuxe6re-reproduserbare-eller-muxe5-de-tvinges}}

Et av vilkårene for pålitelig vitenskapelig arbeid, er reproduserbarhet.
Dette betyr at hvis en annen forsker gjentar undersøkelsen og målingen
ved like betingelser og underliggende faktorer Er det betryggende å få
tilsvarende (like) resultater. En høy reproduserbarhet uttrykkes som
reliabilitet (Bartlett \& Frost, 2008). Det er også viktig å være
oppmerksom på at noen av målingene gir lik resultat, uavhengig av hvor,
når og hvilken forsker som gjennomfører det. For eksempel, kjemiske
analyser i et laboratorium må ha i utgangspunkt høy reproduserbarhet, og
at målinger av studien kan etterprøves. Et annet eksempel er
markedsundersøkelser som utforsker meninger om produkter, tjenester og
kjøpemønstre.

Målinger i sammfunnsvitenskapelig perspektiv innebærer mange
kontekstuelle betingelser, som gjør at resultatet ikke er reproduserbar.
For eksempel måling av hvordan folk reagerer på nyheter i konkrete
politiske atmosfære, eller hvordan velgerne reagerer på kandidater i
politiske kampanjer og hvilke saker som er viktige for dem ved valg.
Dette er noe som ikke er veldig reproduserbart, Forskjere som jobber med
store datasett er nærmest nesten avhengig av å kunne reprodusere det de
jobber med, dette er for å finne ulike avvik og feil i modellen som blir
brukt. Økonomiske modeller som er forsket på må for eksempel kunne bli
reprodusert, likt alle andre kvantitative felt innen forsking. Derfor er
det Forskere har dermed incentiver til å ivareta datasett og kunne
reprodusere modeller i framtiden.

\hypertarget{er-uxf8kt-reproduserbarhet-noe-som-vil-tvinge-seg-fram-eller-er-dagens-uxf8kte-interesse-bare-en-blaff}{%
\subsection{\texorpdfstring{\textbf{Er økt reproduserbarhet noe som vil
tvinge seg fram eller er dagens økte interesse bare en
blaff?}}{Er økt reproduserbarhet noe som vil tvinge seg fram eller er dagens økte interesse bare en blaff?}}\label{er-uxf8kt-reproduserbarhet-noe-som-vil-tvinge-seg-fram-eller-er-dagens-uxf8kte-interesse-bare-en-blaff}}

Økt kunnskapsnivå i sammfunnet, lett tilgjengelighet av datasett,
materiale og utvikling av dataprogrammer og pakker innenfor ulike
programm skreddersydd for hvert felt. Disse ulike faktorene gjør at den
økte interessen vil øke. For eksempel finnes det pakker innenfor Rstudio
som er skreddersydd for psykologer. (Psych) pakken i dette tilfellet.
Alle disse faktorene gjør reproduserbarhet innen forsking og akademiske
journaler viktigere framover.

\hypertarget{hvorfor-er-ikke-flere-studier-reproduserbare-og-replikerbare}{%
\subsection{\texorpdfstring{\textbf{Hvorfor er ikke flere studier
reproduserbare og
replikerbare?}}{Hvorfor er ikke flere studier reproduserbare og replikerbare?}}\label{hvorfor-er-ikke-flere-studier-reproduserbare-og-replikerbare}}

Det kan være ulike grunner til at studier ikke er reproduserbare
og/eller replikerbare. En årsak kan være feilaktig forkastning av null
hypotesen. Dette kan være resultat av feil i analysen eller ett dårlig
utvalg. Det kan også være ett resultat av feilaktig innplotting som har
gått uanmerket. Når analysen blir gjort på ny igjen vil resultatet bli
annerledes. Innen psykologifaget ble det gjort en undersøkelse hvor de
prøvde å replikere 100 undersøkelser fra de siste 3 årene som var blitt
publisert i de tre mest respekterte journalene innenfor faget. De fant
ut att effektene fra funnene i disse studiene ble halvert etter at de
ble gjennomført på ny (replikert).De fant også at i { 95\% } av de
orginale undersøkelsen hadde de fått ett signifikans nivå på 0,05 eller
under, mens etter at undersøkelsen ble utført på ny var det kun { 36\% }
som hadde ett signifikant resultat.

\hypertarget{reproduserbarhet-i-ulike-sektorer}{%
\subsection{\texorpdfstring{\textbf{Reproduserbarhet i ulike
sektorer}}{Reproduserbarhet i ulike sektorer}}\label{reproduserbarhet-i-ulike-sektorer}}

Ulike sektorer er spesielt avhengig av reproduserbarhet, sektor innen
kvantitative og medesin-fagfelt vil være spesielt avhengig av dette, da
avvik og feil i datasettet i form av for eksempel økonomiske modeller
vil koste dem dyrt, de vil være avhengig av en risikomodell som
inkluderer reproduserbarhet. Derfor er de ulike sektorene ulikt avhengig
av reproduserbarhet. Kvalitative felt vil være mindre avhengig av dette.
Da teori ikke alltid er reproduserbar.

\hypertarget{whale-of-london-eksempel}{%
\subsection{\texorpdfstring{\textbf{Whale of London
eksempel}}{Whale of London eksempel}}\label{whale-of-london-eksempel}}

Historien om JP Morgan Chase tabben som kostet selskapet 2 Milliarder
dollar, aksjeinvestorer som jobbet for JP Morgan hadde slått inn en feil
i excel spreadsheet, som underestimerte nedsiden i den syntetiske
modellen som investoren brukte for å kalkulere risikoen i handelen.

Videre undersøkelser viste hvordan tabben en enkel tabbe i excel
spreadsheet remodelerte den finansielle risikomodellen som ble brukt i
sammenheng med investeringen. Derfor er sterkt analytisk,statistikk
rettede programvarer og programmeringspråk som for eksempel ``R'' ofte
overlegen ovenfor excel og analyse via mainstream programmvare som
office suiten til Microsoft. Mange store selskap har i det siste brukt
ressurser på å betale tredjeparter som jobber med datascience, for å
lage modeller til forretningen sin.

\hypertarget{r-markdown-og-r-notebooks}{%
\subsection{\texorpdfstring{\textbf{R-markdown og
R-notebooks}}{R-markdown og R-notebooks}}\label{r-markdown-og-r-notebooks}}

R-Markdown gjør det mulig å skape rike, fullt dokumenterete,
reproduserbare analyser (kilde). R koder, kommentarer, metadata ligger
da inni det samme dokumentet. Dette gjør at man i etterkant kan se
hvilke data som har blitt brukt. Hvilke formler som er brukt for å få
svarene i analysen. Alt som har blitt gjort i analysen kan bli funnet
igjen. Det er ikke gjort analyser separert hvor svarene har blitt
plottet inn i dokumentet i etterkant.Dette dokumentet er skrevet med R
notebook, som er et r-markdown dokument med kode ``chunks'' som kan
utføres uavhengig og interaktivt, hvor resultatet er synlig med en gang.
Dokumentet blir en blanding av vanlig skrift og koder, noe som gjør det
svært effektivt iforhold til andre skriveprogrammer. Kodene som brukes
er ikke synlige i det endelige resultatet som blir publisert, kun
resultatene av disse kodene. Ønsker man derrimot å se hvilke koder som
er blitt brukt er det mulig i etterkant. Dette gjør at dokumentet blir
reproduserbart.

\hypertarget{luxf8ser-r-notebooks-problemet-med-reproduserbarhet}{%
\subsection{\texorpdfstring{\textbf{Løser R notebooks problemet med
reproduserbarhet?}}{Løser R notebooks problemet med reproduserbarhet?}}\label{luxf8ser-r-notebooks-problemet-med-reproduserbarhet}}

Da det er flere ulike programmer,versjoner,pakker og software som blir
brukt til å lagre data og dermed gjøre studier reproduserbare, vil ikke
R-Notebooks på egenhånd løse problemet med reproduserbarhet, det er
derimot et viktig steg og gjør problemet mindre.

\hypertarget{eksempler-puxe5-code-cunktext-chunk-og-sessioninfo-innenfor-r-og-r-notebook}{%
\subsection{\texorpdfstring{\textbf{Eksempler på "Code cunk,Text chunk
og Sessioninfo innenfor R og R
Notebook}}{Eksempler på "Code cunk,Text chunk og Sessioninfo innenfor R og R Notebook}}\label{eksempler-puxe5-code-cunktext-chunk-og-sessioninfo-innenfor-r-og-r-notebook}}

Demonstrasjoner og eksempler på hva ulike verktøy innen R, det finnes
ulike løsninger for alle mulige felt innen forskning og privat bruk.

\hypertarget{text-chunk}{%
\subsubsection{Text Chunk}\label{text-chunk}}

\begin{Shaded}
\begin{Highlighting}[]
\CommentTok{# Her kan man putte tekst i en såkalt "kode-chunk", gjerne for å forklare kodeplotten nedenfor som for eksempel.For å sette inn en code cunk kan man bruke snarveien ctrl+alt+i på windows og cmd+alt+i på OSX.}
\end{Highlighting}
\end{Shaded}

\hypertarget{code-chunk}{%
\subsubsection{Code Chunk}\label{code-chunk}}

\begin{Shaded}
\begin{Highlighting}[]
\KeywordTok{summary}\NormalTok{(iris)}
\end{Highlighting}
\end{Shaded}

\begin{verbatim}
##   Sepal.Length    Sepal.Width     Petal.Length    Petal.Width   
##  Min.   :4.300   Min.   :2.000   Min.   :1.000   Min.   :0.100  
##  1st Qu.:5.100   1st Qu.:2.800   1st Qu.:1.600   1st Qu.:0.300  
##  Median :5.800   Median :3.000   Median :4.350   Median :1.300  
##  Mean   :5.843   Mean   :3.057   Mean   :3.758   Mean   :1.199  
##  3rd Qu.:6.400   3rd Qu.:3.300   3rd Qu.:5.100   3rd Qu.:1.800  
##  Max.   :7.900   Max.   :4.400   Max.   :6.900   Max.   :2.500  
##        Species  
##  setosa    :50  
##  versicolor:50  
##  virginica :50  
##                 
##                 
## 
\end{verbatim}

\begin{Shaded}
\begin{Highlighting}[]
\CommentTok{# Oppsumering av en plot som for eksempel}
\end{Highlighting}
\end{Shaded}

\hypertarget{man-kan-ogsuxe5-legge-inn-plot-og-visualisering-som-for-eksempel}{%
\subsubsection{Man kan også legge inn plot og visualisering som for
eksempel}\label{man-kan-ogsuxe5-legge-inn-plot-og-visualisering-som-for-eksempel}}

\includegraphics{ass1_files/figure-latex/iris-1.pdf}

\hypertarget{sessioninfo-forklart-og-demonstrert}{%
\subsubsection{\texorpdfstring{\textbf{Sessioninfo forklart og
demonstrert}}{Sessioninfo forklart og demonstrert}}\label{sessioninfo-forklart-og-demonstrert}}

Sessioninfo, samler og viser informasjon om den nåværende økten som
kjører i R. Informasjonen inneholder for eksempel, R versjonen som
kjører, operativsystemet som er brukt og de ulike packages som er
lastet. Dette er brukbart for å undersøke mulige feil og avvik, og
bidrar til å gjøre prosjektet mer reproduserbart ved å gi oss mer
informasjon slik at det for eksempel i fremtiden skal bli lett å
reprodusere dette datasettet uavhengig av hva du kjører. Sessioninfo vil
derfor hjelpe med å gjøre dette lettere ved å gi oss mer informasjon om
økten. Nedenfor er dette demonstrert

\begin{Shaded}
\begin{Highlighting}[]
\KeywordTok{sessionInfo}\NormalTok{()}
\end{Highlighting}
\end{Shaded}

\begin{verbatim}
## R version 4.0.2 (2020-06-22)
## Platform: x86_64-apple-darwin17.0 (64-bit)
## Running under: macOS Sierra 10.12.6
## 
## Matrix products: default
## BLAS:   /Library/Frameworks/R.framework/Versions/4.0/Resources/lib/libRblas.dylib
## LAPACK: /Library/Frameworks/R.framework/Versions/4.0/Resources/lib/libRlapack.dylib
## 
## locale:
## [1] en_US.UTF-8/en_US.UTF-8/en_US.UTF-8/C/en_US.UTF-8/en_US.UTF-8
## 
## attached base packages:
## [1] stats     graphics  grDevices utils     datasets  methods   base     
## 
## loaded via a namespace (and not attached):
##  [1] compiler_4.0.2  magrittr_1.5    tools_4.0.2     htmltools_0.5.0
##  [5] yaml_2.2.1      stringi_1.4.6   rmarkdown_2.3   knitr_1.29     
##  [9] stringr_1.4.0   xfun_0.16       digest_0.6.25   rlang_0.4.7    
## [13] evaluate_0.14
\end{verbatim}

\hypertarget{ulike-overskrifter-og-visuelle-skrivemuxe5ter}{%
\subsection{\texorpdfstring{\textbf{Ulike overskrifter og visuelle
skrivemåter}}{Ulike overskrifter og visuelle skrivemåter}}\label{ulike-overskrifter-og-visuelle-skrivemuxe5ter}}

Forskjellige stiler på skrift, demonstrert nedenfor.

\emph{italic}, \textbf{bold} og \textbf{\emph{italic \& bold}}

\emph{HVL Oppgave}, \textbf{HVL Oppgave}, \textbf{\emph{HVL Oppgave}}

Man kan justere størrelse på heading slik, ved å bruke hashtag og antall
i level, demonstrert nedenfor.

\hypertarget{heihei-lvl-3}{%
\subsubsection{Heihei lvl 3}\label{heihei-lvl-3}}

\hypertarget{heihei-lvl-4}{%
\paragraph{Heihei lvl 4}\label{heihei-lvl-4}}

\hypertarget{heihei-lvl-5}{%
\subparagraph{Heihei lvl 5}\label{heihei-lvl-5}}

Heihei lvl 6

Man kan også lage lister,kuleliste, ordnede og uordnene, demonstrert
nedenfor.

\begin{itemize}
\tightlist
\item
  Blå Bil
\item
  Rød Sykkel
\end{itemize}

\hypertarget{konklusjon}{%
\subsection{\texorpdfstring{\textbf{Konklusjon}}{Konklusjon}}\label{konklusjon}}

Reproduserbarhet er nødvendig for å gjennomføre god forsking, vår
konklusjon er at det framover vil blir lettere og viktigere å få til
reproduserbarhet, tilgang på riktig verktøy har blitt lettere i form av
gratis og åpen-kildekode programmvare. Det er også et økende antall
læremateriale lett tilgjengelig på nettet.

Problemet er i form av kompleksitet hvor det er flere versjoner og valg
en kan ta for å få lagret reproduserbar data. Forskere må i framtiden
gjøre data reproduserbar for å få til reproduserbarhet. Enten i form av
peer-review via git eller noe lignende.

\hypertarget{referanser}{%
\subsection{Referanser}\label{referanser}}

\textless DIV id="refs\textless\textgreater/DIV\textgreater{}

\hypertarget{appendiks}{%
\subsection{Appendiks}\label{appendiks}}

\end{document}
